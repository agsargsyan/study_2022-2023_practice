\chapter*{Заключение}
\addcontentsline{toc}{chapter}{Заключение}

За период практики в отделе технической поддержки пользователей
(департамент технологических и информационных ресурсов) РУДН и научных
центрах института прикладной математики и телекоммуникаций.  были
достигнуты все цели и решены все задачи, определенные в программе
научной практики направления подготовки 09.03.03 <<Прикладная
информатика>> программы <<Прикладная информатика>> (см. введение
отчёта по практике). В процессе прохождения практики я работал с
научной терминологией области исследований; научился собирать и
обрабатывать данные, необходимые для формирования
соответствующих выводов исследований; осуществлять целенаправленный
поиск информации на русском и английском языках о новейших научных
достижениях в Интернете и из других источников; строить и анализировать
имитационные модели обьекта исследований.

В результате прохождения данной практики я приобрел следующие
практические навыки, умения, универсальные и профессиональные
компетенции:

\begin{itemize}
\item способность управлять проектом на всех этапах его жизненного
  цикла (постановка задачи, планирование, реализация);
\item способность составлять естесвенно-научные отчеты с IMRAD
  структурой (введение, методы и материалы, результаты и дискуссия);
\item способность разрабатывать имитационные модели и проводить их
  анализ при решении задач в профессиональной области (составлена
  имитационная модель сети с алгоритмом управления очередью на 
  маршрутизаторе типа RED);
\item способность проведения работ по обработке и анализу
  научно-технической информации и результатов исследований (изучение
  необходимой литературы по теме исследования на русском и английском
  языках, подготовка литературного обзора по теме исследований).
\end{itemize}

Таким образом, в рамках практики я рассмотрел моделирование модуля RED
c помощью программного средства NS-2 версии 2.35. Также представлена
программная реализация имитационной модели сети модулем RED и проведен
сравнительный анализ результатов при моделировании сети с разными
входными параметрами, модификаций RED и типов TCP.


