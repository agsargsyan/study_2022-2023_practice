\chapter*{Приложения}
\addcontentsline{toc}{chapter}{Приложения}

%\begin{verbatim}

\begin{minted}[linenos,tabsize=2,breaklines]{bash}
- main.tcl
```
#Создать новый экземпляр объекта Symulator
set ns [new Simulator]
#Открыть трейс файл для nam
set nf [open output/out.nam w]
$ns namtrace-all $nf
set N 20
source "nodes.tcl"
source "links.tcl"
source "queue.tcl"
source "connections.tcl"
source "timing.tcl"
source "nam.tcl"
source "finish.tcl"                                                                                      
#Запуск программы
$ns run
```
- nodes.tcl
```
set node_(r0) [$ns node]
set node_(r1) [$ns node]

for {set i 0} {$i < $N} {incr i} {
        set node_(s$i) [$ns node]
        set node_(s[expr $N + $i]) [$ns node]
        }
```
- links.tcl
```
for {set i 0} {$i < $N} {incr i} {
        $ns duplex-link $node_(s$i) $node_(r0) 100Mb 20ms DropTail
        $ns duplex-link $node_(s[expr $N + $i]) $node_(r1) 100Mb 20ms DropTail
}

$ns simplex-link $node_(r0) $node_(r1) 20Mb 15ms RED
$ns simplex-link $node_(r1) $node_(r0) 15Mb 20ms DropTail
```
- nam.tcl
```
$node_(r0) color "red"
$node_(r1) color "red"
$node_(r0) label "RED"
$node_(r1) shape "square"
$node_(r0) label "square"

$ns simplex-link-op $node_(r0) $node_(r1) orient right
$ns simplex-link-op $node_(r1) $node_(r0) orient left
$ns simplex-link-op $node_(r0) $node_(r1) queuePos 0
$ns simplex-link-op $node_(r1) $node_(r0) queuePos 0

for {set m 0} {$m < $N} {incr m} {
        $ns duplex-link-op $node_(s$m) $node_(r0) orient right
        $ns duplex-link-op $node_(s[expr $N + $m]) $node_(r1) orient left 
}
for {set i 0} {$i < $N} {incr i} {
        $node_(s$i) color "blue"
        $node_(s$i) label "ftp"
}
```
- connections.tcl
```
for {set t 0} {$t < $N} {incr t} {
        $ns color $t green
        set tcp($t) [$ns create-connection TCP/Reno $node_(s$t) TCPSink $node_(s[expr $N + $t]) $t]
        $tcp($t) set window_ 32
        $tcp($t) set maxcwnd_ 32
        set ftp($t) [$tcp($t) attach-source FTP]
}

proc plotWindow {tcpSource file} {
   global ns
   set time 0.01
   set now [$ns now]
   set cwnd [$tcpSource set cwnd_]
   puts $file "$now $cwnd"
   $ns at [expr $now+$time] "plotWindow $tcpSource $file"
}
```
- queue.tcl
```
$ns queue-limit $node_(r0) $node_(r1) 300
$ns queue-limit $node_(r1) $node_(r0) 300

set windowVsTime [open output/WvsT w]
set qmon [$ns monitor-queue $node_(r0) $node_(r1) [open output/qm.out w]]
[$ns link $node_(r0) $node_(r1)] queue-sample-timeout

set redq [[$ns link $node_(r0) $node_(r1)] queue]
$redq set thresh_ 75 #q_min
$redq set maxthresh_ 150  #q_max
$redq set q_weight_ 0.002 # q_weight
$redq set linterm_ 10 # 1/p_max
$redq set gentle_ false  # 
$redq set drop-tail_ true
$redq set queue-in-bytes false #очередь в паетах
$redq set mean_pktsize_ 500
set tchan_ [open output/all.q w]
$redq trace curq_
$redq trace ave_
$redq attach $tchan_
```
- timing.tcl
```
for {set r 0} {$r < $N} {incr r} {
        $ns at 0.0 "$ftp($r) start"
        $ns at 1.0 "plotWindow $tcp($r) $windowVsTime"
        $ns at 24.0 "$ftp($r) stop"
}

$ns at 25.0 "finish"
```
- finish.tcl
```
#Finish procedure
proc finish {} {
   global ns nf
   $ns flush-trace
   close $nf
   global tchan_
   set awkCode {  
      {#запись данных в файлы очереди и средней очереди
         if ($1 == "Q" && NF>2) {
            print $2, $3 >> "output/temp.q";
            set end $2
         }
         else if ($1 == "a" && NF>2)
         print $2, $3 >> "output/temp.a";
      }
   }

   set f [open output/temp.queue w]
   puts $f "TitleText: RED"
   puts $f "Device: Postscript"

   if { [info exists tchan_] } {
      close $tchan_
   }
   #обновление данных
   exec rm -f output/temp.q output/temp.a
   exec touch output/temp.a output/temp.q

   exec awk $awkCode output/all.q

   puts $f \"queue
   exec cat output/temp.q >@ $f
   puts $f \n\"ave_queue
   exec cat output/temp.a >@ $f
   close $f
   # вывод в xgraph
   exec xgraph -bb -tk -x time -t "TCPRenoCWND" output/WvsT &
   exec xgraph -bb -tk -x time -y queue output/temp.queue &
   exit 0
}
```
- out.gp
```
#!/usr/bin/gnuplot -persist

#вывод графиков
set terminal postscript eps
set output "images/queues.eps"
set xlabel "Time, s"
set ylabel "Queue Length [pkt]"
set title "RED Queue"
plot "output/temp.q" with lines linestyle 1 lt 1 lw 2 title "Queue length", "output/temp.a" with lines linestyle 2 lt 3 lw 2 title "Average queue length"

set terminal postscript eps
set output "images/TCP.eps"
set xlabel "Time (s)"
set ylabel "Window size [pkt]"
set title "TCPVsWindow"
plot "output/WvsT" with lines linestyle 1 lt 1 lw 2 title "WvsT"
```
\end{minted}
% \end{verbatim}


