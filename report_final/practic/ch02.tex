\chapter{О Vanet }

\section{Определение VANET}


Интеллектуальные транспортные сети~\cite{AutonomousVehicles,VehicularNetworking}, также известные как VANETs,
представляют собой беспроводные коммуникационные сети, образованные
транспортными средствами и интеллектуальной транспортной
инфраструктурой, такой как светофоры и знаки.  VANETs предназначены
для повышения безопасности дорожного движения, управления дорожным
движением и эффективности движения за счет обеспечения прямой связи
между транспортными средствами и интеллектуальной транспортной
инфраструктурой. VANET — это развивающаяся технология со значительным
потенциалом для преобразования способов вождения и взаимодействия на
дорогах.

\section{Технология VANET}

VANETs основаны на технологиях беспроводной связи, таких как
радиосвязь малого радиуса действия (SRRC), которые позволяют
транспортным средствам напрямую общаться друг с другом и с
интеллектуальной транспортной инфраструктурой. Транспортные средства
могут обмениваться информацией о своем местоположении, скорости,
направлении движения и другими данными, важными для безопасности
дорожного движения. Интеллектуальная транспортная инфраструктура также
может взаимодействовать с транспортными средствами для предоставления
информации о дорожных условиях, светофорах, знаках, дорожных работах и
авариях.

VANET предлагают ряд преимуществ, включая повышение безопасности
дорожного движения за счет обеспечения прямой связи между
транспортными средствами, повышение эффективности дорожного движения
за счет предоставления информации о дорожных условиях в режиме
реального времени, улучшение управления дорожным движением за счет
обеспечения связи инфраструктуры ИТС с транспортными средствами и
улучшение связи в чрезвычайных с ситуациях за счет предоставления
точной информации о дорожных условиях аварийным службам.

Однако VANET имеет и недостатки, такие как проблемы безопасности,
высокая стоимость и проблемы совместимости.  Безопасность является
важным вопросом, поскольку VANET уязвима для злонамеренных атак, а
нарушение безопасности может привести к авариям или нарушению
дорожного движения.

Преимущества VANET:
\begin{itemize}
\item повышение безопасности дорожного движения: VANET может повысить
  безопасность дорожного движения, обеспечивая прямую связь между
  транспортными средствами, что может помочь предотвратить аварии в
  режиме реального времени;
\item повышение эффективности дорожного движения: VANET может помочь
  уменьшить дорожные заторы, предоставляя информацию о дорожных
  условиях в режиме реального времени, что может помочь водителям
  принимать более обоснованные решения;
\item улучшенное управление дорожным движением: VANET может обеспечить
  более эффективное управление дорожным движением, позволяя
  интеллектуальным транспортным инфраструктурам общаться с
  транспортными средствами, что может помочь регулировать дорожное
  движение; 
\item улучшение связи в чрезвычайных ситуациях: VANET может помочь
  аварийным службам быстрее добраться до места аварии, предоставляя
  точную информацию о дорожных условиях.
\end{itemize}

%Однако есть и недостатки, которые необходимо учитывать.

Основные недостатки VANET:
\begin{itemize}
\item Безопасность: как и для всех беспроводных технологий,
  безопасность VANETs является важным вопросом. VANETs уязвимы для
  злонамеренных атак, и нарушение безопасности может привести к
  авариям или нарушению движения.
\item Стоимость: VANETs требуют значительных инвестиций в аппаратное и
  программное обеспечение. Это может сделать технологию непомерно
  дорогой для некоторых регионов или стран.
\item Совместимость: VANETs могут быть построены с использованием
  различных типов беспроводных технологий, что может сделать
  совместимость сложной задачей.
 \end{itemize}

 В реальной жизни развёртывание VANET обычно осуществляется
 технологическими компаниями, специализирующимися на транспортных
 системах связи. Транспортные средства могут быть оснащены
 устройствами беспроводной связи, такими как модемы или
 приемопередатчики, которые позволяют им подключаться к сети
 VANET. Интеллектуальная транспортная инфраструктура, такая как
 светофоры и знаки, также оснащается беспроводными коммуникационными
 устройствами для подключения к сети VANET. Протоколы связи
 разработаны таким образом, чтобы обеспечить двустороннюю связь между
 транспортными средствами и интеллектуальной транспортной
 инфраструктурой.

 \section{О IEEE 802.11p}


 Стандарт IEEE 802.11p является важным достижением в области
 интеллектуальных сетей для транспортных средств. Обеспечивая
 своевременную и надежную связь между транспортными средствами и
 интеллектуальной транспортной инфраструктурой, он прокладывает путь к
 более эффективным приложениям для обеспечения безопасности дорожного
 движения и управления трафиком.

 Стандарт использует радиочастоты 5,9 ГГц и технологию модуляции с
 распределенным спектром для повышения устойчивости к помехам и
 обеспечения когерентной связи даже в шумной среде. Он также
 предоставляет функции качества обслуживания (QoS) для обеспечения
 надежной связи в условиях высокой плотности трафика.

 Области применения IEEE 802.11p многочисленны. Он может
 использоваться для связи между транспортными средствами (V2V) и между
 транспортными средствами и инфраструктурой (V2I), управления дорожным
 движением и помощи водителю. Например, связь V2V может помочь
 предотвратить аварии, позволяя транспортным средствам сигнализировать
 друг другу, когда они приближаются слишком близко. Связь V2I может
 помочь регулировать дорожное движение в режиме реального времени,
 информируя водителей о предстоящих дорожных условиях.

 Однако стандарт IEEE 802.11p не лишен трудностей. Для его
 эффективности необходима соответствующая инфраструктура связи,
 которая может быть дорогостоящей. Кроме того, он должен быть способен
 обрабатывать большие объемы данных, генерируемых подключенными
 автомобилями, что может создать проблемы с пропускной способностью и
 вычислительной мощностью.



 802.11p --- это стандарт IEEE, определяющий беспроводную связь для
 интеллектуальных транспортных систем (ITS) и подключенных
 транспортных средств.

%  Ниже приведена общая методология реализации
%  решения 802.11p:

%  Определите функциональные и нефункциональные требования вашего
%  проекта. Сюда входит количество транспортных средств, дальность
%  связи, скорость передачи данных, безопасность, задержка и т.д.

% Выберите аппаратные и программные компоненты, необходимые для решения
% 802.11p. Сюда входят приемопередатчики 802.11p, процессоры, антенны,
% драйверы устройств, средства разработки программного обеспечения и
% т.д.

% Разработайте архитектуру сети 802.11p. Сюда входит выбор топологии
% сети (например, ячейка, звезда), адресация и управление безопасностью,
% определение каналов связи и т.д.

% Программирование узлов 802.11p. Сюда входит реализация протокола 802.11p на каждом узле, управление конфигурацией и обновлением узлов, управление качеством обслуживания и т.д.

% Тестирование системы 802.11p. Сюда входит создание тестовой среды для
% оценки производительности и стабильности системы, проверка
% соответствия стандартам 802.11p, обнаружение и исправление ошибок и
% т.д.

% Развертывание сети 802.11p. Это включает в себя настройку узлов сети,
% физическую установку узлов, ввод в эксплуатацию и текущее обслуживание
% сети и т.д.


% \section{заключение}

% Автомобильные интеллектуальные сети (VANET) являются развивающейся
% технологией со значительным потенциалом для повышения безопасности
% дорожного движения, управления движением и эффективности
% движения. VANET основана на технологиях беспроводной связи и позволяют
% осуществлять прямую связь между транспортными редствами и
% интеллектуальной транспортной инфраструктурой. Хотя VANET обладают
% значительными преимуществами, они также имеют недостатки, такие как
% вопросы безопасности, высокая стоимость и проблемы
% совместимости. VANET являются быстро развивающейся технологией, и их
% внедрение будет зависеть от способности преодолеть эти проблемы и
% разработать инновационные решения для повышения безопасности и
% эффективности дорожной сети.













